<p style="display:none;"> % Tag useful such that part is included but already hidden even before rendering of MathJax.
\(
\newcommand{\bm}[1]{\boldsymbol{#1}}
% \newcommand{\bm}[1]{\boldsymbol{\mathbf{#1}}}  % Use this if you prefere upright (not italic) vectors and matrices

% Operators
\newcommand{\tr}{{}^{\intercal}}               % transpose (needs amssymb)
\newcommand{\inv}{^{-1}}                       % inverse
\renewcommand{\d}[1]{\operatorname{d}\!{#1}}  % dx
\newcommand{\norm}[1]{\left\lVert#1\right\rVert}           % L2 Norm
\DeclareMathOperator*{\argmin}{arg\, min}
\DeclareMathOperator*{\argmax}{arg\, max}
\DeclareMathOperator{\diag}{diag}

\newcommand{\E}{\mathbf{E}}  % Depreciated: delete as soon as possible... Use \ERW{}

% Matrix
\newcommand{\bigzero}{\mbox{\normalfont\Large\bfseries 0}}  % big zero in matrix
\newcommand{\rvline}{\hspace*{-\arraycolsep}\vline\hspace*{-\arraycolsep}} % vertical bar to separate matrix

% SfS stuff
\newcommand{\iid}{\mbox{ i.i.d. }}
\newcommand{\Xsub}[2]{{#1}_{\mathrm{\scriptscriptstyle #2}}}
\newcommand{\wh}[1]{{}\kern0.1em\widehat{\kern-0.1em#1}{}}

\DeclareMathOperator{\Var}{Var}
\DeclareMathOperator{\Cov}{Cov}
\DeclareMathOperator{\Cor}{Corr}

\newcommand{\ekl}{[}     % Erwartungswert Klammer links
\newcommand{\ekr}{]}     % Erwartungswert Klammer rechts
\newcommand{\ckl}{(}     % (Co)variance Klammer links
\newcommand{\ckr}{)}
\newcommand{\pkl}{(}     % Probability Klammer links
\newcommand{\pkr}{)}
\newcommand{\vkl}{(}     % Verteilungs Klammer links
\newcommand{\vkr}{)}

\newcommand{\ERWSymbol}{\mathbf{E}}  % vorbereitet zum Anpassen an individuelle Styles
\newcommand{\VARSymbol}{\Var}
\newcommand{\COVSymbol}{\Cov}
\newcommand{\CORSymbol}{\Cor}
\DeclareMathOperator{\PRSymbol}{P}
\newcommand{\NormalSymbol}{\mathcal{N}}
\DeclareMathOperator{\BinomialSymbol}{\mathcal{B}} % Binomial
\DeclareMathOperator{\BernoulliSymbol}{\mathcal{B}ernoulli} % Bernoulli
\DeclareMathOperator{\ExponentialSymbol}{Exp}
\DeclareMathOperator{\PoissonSymbol}{Pois}

% die eigentlichen ``Funktionen''
\newcommand{\ERW}[1] {\ERWSymbol \left[ #1 \right]}
\newcommand{\VAR}[1] {\VARSymbol \left( #1 \right)}
\newcommand{\VARH}[1]{\wh\VARSymbol \left( #1 \right)}

\newcommand{\COV}[1] {\COVSymbol \left(#1 \right)}
\newcommand{\COVH}[1] {\wh\COVSymbol \left( #1 \right)}
\newcommand{\COR}[1] {\CORSymbol    \left( #1 \right)}
\newcommand{\CORH}[1] {\wh\CORSymbol \left( #1 \right)}
\newcommand{\PR}[1] {\PRSymbol \! \left( #1 \right)}
\newcommand{\Normal}[1] {\NormalSymbol   \left( #1 \right)}
\newcommand{\Binomial}[1] {\BinomialSymbol    \left( #1 \right)}
\newcommand{\Bernoulli}[1] {\BernoulliSymbol    \left( #1 \right)}
\newcommand{\Exponential}[1] {\ExponentialSymbol    \left( #1 \right)}
\newcommand{\Poisson}[1] {\PoissonSymbol   \left( #1 \right)}

\newcommand{\ERWi}[2]  {\Xsub{\ERWSymbol}{#1}    \left[ #2
    \right]}
\newcommand{\VARi}[2]  {\Xsub{\VARSymbol}{#1}    \left( #2
    \right)}
\newcommand{\VARHi}[2] {\Xsub{\wh\VARSymbol}{#1} \left( #2
    \right)}
\newcommand{\COVi}[2]  {\Xsub{\COVSymbol}{#1}    \left( #2
    \right)}
\newcommand{\COVHi}[2] {\Xsub{\wh\COVSymbol}{#1} \left( #2
    \right)}
\newcommand{\CORi}[2]  {\Xsub{\CORSymbol}{#1}    \left( #2
    \right)}
\newcommand{\CORHi}[2] {\Xsub{\wh\CORSymbol}{#1} \left( #2
    \right)}
\newcommand{\PRi}[2]   {\Xsub{\PRSymbol}{#1}     \left( #2
    \right)}
\newcommand{\Normali}[2]      {\Xsub{\NormalSymbol}{#1}   \left( #2
    \right)}
\newcommand{\Binomiali}[2]    {\Xsub{\BinomialSymbol}{#1}    \left( #2
    \right)}
\newcommand{\Bernoullii}[2]   {\Xsub{\BernoulliSymbol}{#1}    \left( #2
    \right)}
\newcommand{\Exponentiali}[2] {\Xsub{\ExponentialSymbol}{#1}    \left( #2
    \right)}
\newcommand{\Poissoni}[2]     {\Xsub{\PoissonSymbol}{#1}   \left( #2
    \right)}
\)
</p>